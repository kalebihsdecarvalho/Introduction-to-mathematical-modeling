\documentclass{article}
\usepackage{amsmath, amssymb}
\usepackage{graphicx} % Required for inserting images
\usepackage{minted}

\title{Deep Learning - Introdução a Modelagem Matemática}
\author{}
\date{}

\begin{document}

\maketitle

\section*{10.1 Exercício: Rede MLP - Multilayer Perceptron}

Considere o conjunto de dados abaixo constituído pelos dados de entrada nas três primeiras colunas (x1, x2, x3) da matriz seguido pelo respectivo rótulo (0, 1 ou 2).


\begin{center}
    \begin{minted}{python}
    #          x1            x2          x3     classe
dataset = [[ 1.02153588, -1.04584554, -0.96943922,  0], 
           [ 1.07472359, -0.81372418, -0.50227571,  0], 
           [ 1.03845087, -0.85529440, -1.07551718,  0], 
           [ 1.08671323, -0.39041877, -1.06912290,  0], 
           [ 1.81386050, -1.03705351, -1.01491796,  0], 
           [ 1.60285424, -0.53666876, -0.68868644,  0], 
           [ 1.67743150, -0.68302534, -0.58474262,  0], 
           [ 1.34016832, -0.33036141, -0.45972834,  0], 
           [ 1.15715279, -0.57687815, -0.86905314,  0], 
           [ 1.19312552, -0.74858765, -0.26682942,  0], 
           [ 1.32535135, -0.51619190, -0.50434856,  0], 
           [ 1.17080360, -0.29582611, -0.31267014,  0], 
           [-0.60098243,  1.13155099, -0.77290891,  1], 
           [-0.78660704,  1.22050116, -1.05339234,  1], 
           [-0.42366120,  1.29704384, -0.94234171,  1], 
           [-0.66885149,  1.43052978, -1.00206341,  1], 
           [-0.46490586,  1.76682434, -0.74631286,  1], 
           [-1.02489359,  1.35547338, -0.38458331,  1], 
           [-0.96717853,  1.46557232, -0.68402367,  1], 
           [-0.57114584,  1.15404176, -0.87468506,  1], 
           [-0.94714779,  1.13832305, -0.58694270,  1], 
           [-0.40102286,  1.46159431, -0.69792237,  1], 
           [-0.26437944,  1.34796154, -0.73774277,  1], 
           [-0.27293769,  1.59309487, -1.04274151,  1], 
           [-0.30283821, -0.73600306,  1.44039980,  2], 
           [-0.58290259, -0.25539981,  1.50781368,  2], 
           [-1.08018241, -0.25738737,  1.09149484,  2], 
           [-0.76039084, -0.97361097,  1.28860632,  2], 
           [-1.04975329, -0.77085457,  1.78097885,  2], 
           [-0.40436016, -0.52396243,  1.51026685,  2], 
           [-0.83580163, -0.84298958,  1.05583722,  2], 
           [-0.84797325, -0.47850486,  1.55482311,  2], 
           [-0.27698582, -0.67465935,  1.74025219,  2], 
           [-0.94195261, -0.72946186,  1.79217650,  2], 
           [-0.48707297, -0.86887812,  1.20341262,  2], 
           [-0.82242448, -0.75027166,  1.49045897,  2]]
    \end{minted}
\end{center}

Implemente o treinamento supervisionado de uma rede MLP considerando todo o conjunto de dados acima como conjunto de treinamento. 

\textbf{Considerações:}
\begin{itemize}
    \item O modelo deve possuir uma camada com quatro neurônios recebendo os dados de entrada (x1, x2, x3 e o bias) seguida pela camada de saída com três neurônios (um neurônio para cada classe ) recebendo as saídas da camada escondida;

    \item Utilize a técnica "One-Hot-Encoding" parta converter os rótulos de saída

    \item  Repita o laço de treinamento 20 vezes

    \item  Utilize 
    \begin{center}
    \begin{minted}{python}
        LEARNING_RATE = 0.01
        np.random.seed(7)
    \end{minted}
\end{center}
\end{itemize}

\textbf{Entrada:} Nova instância (x1, x2, x3) para classificar.

\textbf{Saída:} A classe da instância de entrada. Por exemplo, a saída dos neurônios do primeiro caso teste é:

\begin{center}
    \begin{minted}{python}
        [0.46114314 0.4329263  0.43107695]
    \end{minted}
\end{center}

porém a classe é dada pela posição do maior valor, ou seja:

\begin{center}
    \begin{minted}{python}
       output_layer_y.argmax()
    \end{minted}
\end{center}

resultando na classe:
\begin{center}
    \begin{minted}{python}
       >> 0
    \end{minted}
\end{center}
\end{document}